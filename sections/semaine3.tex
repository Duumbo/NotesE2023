\documentclass{subfiles}[../main.tex]

\begin{document}
    \section{Semaine 3} % (fold)
    \label{sec:Semaine 3}
        \subsection{Lundi le 15 Mai} % (fold)
        \label{sub:Lundi le 15 Mai}

            \subsubsection{Exemple hamiltonien}
            \label{sec:Exemple hamiltonien}
                Pour diagonaliser l'hamiltonien, on veut trouver les sous-espaces
                qui sont indépendants. On cherche donc à trouver les blocs de
                l'hamiltonien. Pour ce faire, il faut encoder l'application de
                $H$ sur nos états. Rappelons l'hamiltonien de Hubbard:
                \begin{align}
                    H&=U\sum_in_{i\uparrow}n_{i\downarrow}-
                    \sum_{<i,j>,\sigma}t_{ij}\qty(c^\dagger_{i\sigma}
                    c_{j\sigma}+c^\dagger_{j\sigma}
                    c_{i\sigma})
                \end{align}
                La base de l'espace de Fock qu'on utilise sera l'ordre normal
                énoncé plus haut. Alors on peut définir l'application des
                opérateurs qui composent $H$ sur les vecteurs de la base.
                \begin{align}
                    c^\dagger_{i\sigma}\ket{n}&=
                    \begin{cases}
                        \texttt{xor } n, 00\cdots1\cdots00, &\text{ à la position }
                        i\sigma, \text{si le bit } i\sigma \text{est nul.}\\
                        0000 & \text{autrement.}
                    \end{cases}
                \end{align}
                Or, cette notation peut mélanger le vecteur nul et l'état
                $\ket{0}$. On peut régler ce problème en introduisant un bit de
                signe. Alors tous les états sont représentés avec un $1$ préposant
                tous les autres bits, le bit nul sera alors toujours le vecteur
                nul.\\
                Les opérateurs nombres sont simplement un \texttt{and} avec
                le bitshift adéquat. Par exemple, si je veut calculer
                $\bra{n}n_{i\uparrow}n_{i\downarrow}\ket{n}$, il faut
                \begin{figure}[H]
                \begin{minted}[linenos,frame=single]{nasm}
MOV EAX, n
MOV EBX, n
SHL EAX, N
AND EAX, EBX
                \end{minted}
                    \caption{Pseudo-assembly représentant comment appliquer
                    l'opérateur $n_{i\uparrow}n_{i\downarrow}$, pour tous les
                    $i$. La valeur $n$ représente l'état de Fock dans l'ordre
                    normal, tandis que $N$ représente le nombre de site. Le
                    résultat attendu se trouvera dans le registre \textbf{EAX},
                    dans la première moitié du byte,
                    pour les sites de $i=N-1$ jusqu'à $i=0$.}
                \end{figure}
                Cette implémentation du terme de l'hamiltonien a quelque
                limitation qui sont facilement corrigeable. En effet, supposons
                que nous avons les états $\ket{7}$ et $\ket{8}$, alors dépendant
                quel état que l'on shifte à gauche, on n'aura pas le même résultat.
                Mais ceci n'est pas important, car cet opérateur est diagonal
                dans notre base, alors on peut simplement juste calculer cet
                élément de matrice pour les éléments diagonaux.\\
                Le produit scalaire des états s'implémente facilement avec des
                opérations sur des nombres binaires, $\braket{n'}{n}$ s'écrit
                \begin{figure}[H]
                \begin{minted}[linenos,frame=single]{nasm}
MOV EAX n'
MOV EBX n
XOR EAX EBX
                \end{minted}
                    \caption{Produit scalaire en $n$ et $n'$. Cette valeur sera
                    $1$ si \textbf{EAX} est nul, $0$ pour toute autre valeur de
                    \textbf{EAX}.}
                \end{figure}
                Le plus difficile est d'implémenter l'application de l'opérateur.
                $(c^\dagger_ic_j+c^\dagger_jc_i)\ket{n'}$, pour tous les spins. Dans ce
                cas, il y a des conditions que je ne pense pas être capable de
                retirer.
                Commençons par penser à quels sites qu'on a le droit de détruire.
                Soit la bitstring $n_i$, alors les positions que j'ai le droit
                de créer sont $\neg n_i$. Les positions que j'ai le droit de
                détruire sont $n_i$. Ainsi, les positions que j'ai le droit
                de détruire puis créer sont $(\neg n_j)\&n_i$.
            % subsubsection Exemple hamiltonien (end)
        % subsection Lundi le 15 Mai (end)

    % section Semaine 3 (end)
\end{document}
