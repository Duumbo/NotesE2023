\documentclass{subfiles}[../main.tex]

\begin{document}
    \section{Semaine 2} % (fold)
    \label{sec:Semaine 2}
        \subsection{Mardi le 9 Mai} % (fold)
        \label{sub:Mardi le 9 Mai}

            % Journée du mardi le 9 mai
            \subsubsection{Changement de base} % (fold)
            \label{sec:Changement de base}
            Soit effectuons le changement de base de l'Hamiltonien de Hubbard.
            \begin{align}
                H=U\sum_{i}n_{i\uparrow}n_{i\downarrow}
                -\sum_{<i,j>,\sigma}t_{ij}\qty(
                    c^\dagger_{i,\sigma}c_{j,\sigma}
                    +
                    c^\dagger_{j,\sigma}c_{i,\sigma}
                )
            \end{align}
            Les transformées de Fourrier des opérateurs création et
            annihilation sont
            \begin{align}
                c^\dagger_{i,\bm{k}}&=\frac{1}{\sqrt{N}}\sum_{\bm{r}}
                e^{i\bm{k}\cdot\bm{r}}
                c^\dagger_{i,\bm{r}}\\
                c^\dagger_{i,\bm{r}}&=\frac{1}{\sqrt{N}}\sum_{\bm{k}}
                e^{-i\bm{k}\cdot\bm{r}}
                c^\dagger_{i,\bm{r}}
            \end{align}
            Allons-y terme à terme.

            % subsubsection Changement de base (end)
            \subsubsection{Hubbard Intro Old} % (fold)
            \label{sec:Hubbard Intro Old}
            % Problème 1:
            \begin{problem}
                Supposons que $N=2$.
            Construire une représentation matricielle
            explicite, dans l'espace de Hilbert global
            de dimension $2^2=4$, des opérateurs suivants:
            $c_1,c_2,n_1,n_2$
            \end{problem}
            Soit la base de l'espace de Hilbert
            \begin{align}
                \left\{
                \ket{00},\ket{01},\ket{10},\ket{11}
                    \right\}
            \end{align}
            On sait que $c_1$ détruit le premier site, tandis que $c_2$ détruit
            le second, on peut donc les écrires en terme de mapping entre les
            états
            \begin{align}
                c_1:\quad\left\{
                    \ket{00}\rightarrow\bm{0},
                    \ket{01}\rightarrow\ket{00},
                    \ket{10}\rightarrow\bm{0},
                    \ket{11}\rightarrow\ket{10}
                    \right\}\\
                c_2:\quad\left\{
                    \ket{00}\rightarrow\bm{0},
                    \ket{01}\rightarrow\bm{0},
                    \ket{10}\rightarrow\ket{00},
                    \ket{11}\rightarrow\ket{01}
                    \right\}
            \end{align}
            Et donc si on écrit les états suivants
            \begin{align}
                \ket{00}&=\begin{pmatrix}
                    1\\0\\0\\0
                \end{pmatrix}\\
                \ket{01}&=\begin{pmatrix}
                    0\\1\\0\\0
                \end{pmatrix}\\
                \ket{10}&=\begin{pmatrix}
                    0\\0\\1\\0
                \end{pmatrix}\\
                \ket{11}&=\begin{pmatrix}
                    0\\0\\0\\1
                \end{pmatrix}
            \end{align}
            On obtient les matrices
            \begin{align}
                c_1&=\begin{pmatrix}
                    0&1&0&0\\
                    0&0&0&0\\
                    0&0&0&1\\
                    0&0&0&0
                \end{pmatrix}\\
                c_2&=\begin{pmatrix}
                    0&0&1&0\\
                    0&0&0&1\\
                    0&0&0&0\\
                    0&0&0&0
                \end{pmatrix}
            \end{align}

            % Problème 2:
            \begin{problem}
                Supposons que $N=2$ et
            considérons un état général à un électron
            $\ket{\psi}=\alpha\ket{1}+\beta\ket{2}$,
            où $\abs{\alpha}^2+\abs{\beta}^2=1$. Quelle
            est la probabilité que l'électron soit sur
            l'ion no $1$? Quelles sont les valeurs moyennes
            de $n_1$ et $n_2$?
            \end{problem}
            La probabilité d'être sur le site $1$ est
            \begin{align}
                \abs{\braket{1}{\psi}}^2&=\abs{\alpha}^2
            \end{align}
            Les valeurs moyennes sont
            \begin{align}
                \ket{1}=\ket{01}\quad&\quad\ket{2}=\ket{10}\\
                \bra{\psi}n_1\ket{\psi}&=\abs{\alpha}^2\\
                \bra{\psi}n_2\ket{\psi}&=\abs{\beta}^2
            \end{align}

            % Problème 3:
            \begin{problem}
                Considérer un modèle
            à $N=4$ sites et comportant $M=4$ électrons,
            avec $t=0$. Trouver les deux niveaux d'énergie
            les plus bas avec les états correspondants.
            Combien y en a-t-il pour chacun des deux
            niveaux?
            \end{problem}
            Pour $t-0$, le Hamiltonien de Hubbard devient
            \begin{align}
                H=V&=U\sum_{i}n_{i\uparrow}n_{i\downarrow}
            \end{align}
            Avec $M=3$ électrons, le fondamental est celui que les trois
            électrons sont sur des sites différents, dégénérés plusieurs fois
            avec une énergie de $E=0$. Le premier état excité est celui qui
            a un site avec deux électrons de spins opposés, avec une énergie de
            $E=U$, $4$ fois dégénéré.

            % Problème 4:
            \begin{problem}
                Démontrer la relation 22
            et ensuite que les deux équations de 20 sont
            compatibles.
            \end{problem}
            Commençons par $r=0$, soit
            \begin{align}
                \sum_{k}e^{ikr}=\sum_k 1=N
            \end{align}
            Ensuite, allons-y avec $r\neq 0$.
            \begin{align}
                \sum_{j=0}^{N-1}e^{ir(2\pi j/N)}&=\frac{1-e^{2\pi ir}}
                {1-e^{2\pi ir/N}}
            \end{align}
            Comme $r$ est un point du réseau, alors il s'agit d'un entier. Le
            numérateur s'annule donc, ce qu'il fallait démontrer.

            % Problème 5:
            \begin{problem}
                Démontrer que les opérateurs
                $\widetilde{c_\sigma}(\bm{k})$ et
                $\widetilde{c_\sigma}(\bm{k})$ satisfont
                aux relations d'anticommutation suivantes:
                \begin{align}
                    &\{\widetilde{c}_\sigma(\bm{k}),
                    \widetilde{c}_{\sigma'}(\bm{k'})\}=0
                    &&\{\widetilde{c}_\sigma(\bm{k}),
                    \widetilde{c}_{\sigma'}^\dagger(\bm{k'}
                    )\}=\delta_{\sigma\sigma'}
                    \delta_{\bm{k}\bm{k'}}
                \end{align}
            \end{problem}
            Partons des relations d'anti-commutation des opérateurs définis
            précédemment
            \begin{align}
                &\{c_{i\sigma},c_{j,\sigma'}\}=0
                &&\{c^\dagger_{i\sigma},c^\dagger_{j,\sigma'}\}=0
                &&\{c_{i\sigma},c^\dagger_{j,\sigma'}\}=\delta_{ij}
                \delta_{\sigma\sigma'}
            \end{align}
            Soit alors les transformations de Fourier
            \begin{align}
                &\widetilde{c}_{\sigma}(k)=\frac{1}{\sqrt{N}}\sum_re^{-ikr}
                c_{\sigma}(r)
                &\widetilde{c}^\dagger_{\sigma}(k)=\frac{1}{\sqrt{N}}\sum_re^{ikr}
                c^\dagger_{\sigma}(r)
            \end{align}
            Alors, comme l'anti-commutateur est bilinéaire, on a que
            \begin{align}
                \{\widetilde{c}_\sigma(\bm{k}),
                    \widetilde{c}_{\sigma'}(\bm{k'})\}
                    =\frac1N\sum_{r,r'}e^{-irk-ir'k'}\{c_{\sigma}(r),
                    c_{\sigma'}(r')\}=0\\
                \{\widetilde{c}^\dagger_\sigma(\bm{k}),
                    \widetilde{c}^\dagger_{\sigma'}(\bm{k'})\}
                    =\frac1N\sum_{r,r'}e^{irk+ir'k'}\{c^\dagger_{\sigma}(r),
                    c^\dagger_{\sigma'}(r')\}=0\\
                \{\widetilde{c}_\sigma(\bm{k}),
                    \widetilde{c}^\dagger_{\sigma'}(\bm{k'})\}
                    =\frac1N\sum_{r,r'}e^{-irk+ir'k'}\{c_{\sigma}(r),
                    c^\dagger_{\sigma'}(r')\}=\frac1N\sum_re^{-ir(k-k')}
                    \delta_{\sigma\sigma'}\\
                    =\delta_{kk'}\delta_{\sigma\sigma'}
            \end{align}

            %Problème 6:
            \begin{problem}
                Démontrer que l'opérateur $K$ s'exprime
                comme suit en fonction des opérateurs
                $\widetilde{c_\sigma}(\bm{k})$ et
                $\widetilde{c_\sigma}^\dagger(\bm{k})$:
                \begin{align}
                    &K=-2t\sum_{\bm{k}\sigma}
                    \cos(k)n_\sigma(\bm{k})
                    &&n_\sigma(\bm{k})\equiv
                    \widetilde{c}_\sigma^\dagger(\bm{k})
                    \widetilde{c}_\sigma(\bm{k})
                \end{align}
            \end{problem}
            Soit l'opérateur $K$
            \begin{align}
                K&=-t\sum_{<i,j>,\sigma}\qty(c^\dagger_{i,\sigma}c_{j,\sigma}
                +c^\dagger_{j,\sigma}c_{i,\sigma})\\
                &=-t\sum_{R,r,\sigma}\qty(c^\dagger_{\sigma}(R+r)c_{\sigma}(R-r)
                +c^\dagger_{\sigma}(R-r)c_{\sigma}(R+r))
            \end{align}
            avec
            \begin{align}
                &R=\frac{r_1+r_2}{2}
                &&r=\frac{r_1-r_2}{2}
            \end{align}
            Alors
            \begin{align}
                K&=-\frac{t}{N}\sum_{R,r,\sigma,k,k'}\qty(
                    c^\dagger_{\sigma}(k)c_\sigma(k')e^{i(R+r)k-i(R-r)k'}+
                    c^\dagger_{\sigma}(k')c_\sigma(k)e^{i(R-r)k-i(R+r)k'}
                )\\
                &=-\frac{t}{N}\sum_{R,r,\sigma,k,k'}\qty(
                    c^\dagger_{\sigma}(k)c_\sigma(k')e^{ir(k+k')+iR(k-k')}+
                    c^\dagger_{\sigma}(k')c_\sigma(k)e^{-ir(k+k')+iR(k-k')}
                )\\
                &=-t\sum_{r,\sigma,k}\qty(
                    c^\dagger_{\sigma}(k)c_\sigma(k)e^{i2rk}+
                    c^\dagger_{\sigma}(k)c_\sigma(k)e^{-2irk}
                )\\
                &=-2t\sum_{r,\sigma,k}
                    c^\dagger_{\sigma}(k)c_\sigma(k)\cos(2kr)
            \end{align}

            %Problème 7:
            \begin{problem}
                Démontrer que les opérateurs de nombre
                $n_\sigma(\bm{k})$ associés à des spins ou
                des nombres d'ondes différents commutent.
            \end{problem}
            Les opérateurs création et annihilation associés à des nombres
            quantiques différents
            anticommutent. Alors, comme l'opérateur
            nombre est composé de deux opérateurs création et annihilation,
            il faut utiliser deux fois les relations d'anti-commutation pour
            faire traverser un autre opérateur création annihilation. Par
            conséquent, les opérateurs nombres et les opérateurs création
            annihilation associés à des nombres quantiques différents commutent.
            Il en va de soi que les opérateurs nombre associés à des nombres
            quantiques différents commutent.

            %Problème 8:
            \begin{problem}
                Quel est l'état fondamental d'un système de
                $N$ électrons installés sur un anneau de
                $N$ sites? On est à demi-remplissage, car

            \end{problem}

            % subsubsection Hubbard Intro Old (end)


        % subsection Mardi le 9 Mai (end)

    % section Semaine 2 (end)
\end{document}
