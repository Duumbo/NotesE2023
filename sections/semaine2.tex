\documentclass{subfiles}[../main.tex]

\begin{document}
    \section{Semaine 2} % (fold)
    \label{sec:Semaine 2}
        \subsection{Mardi le 9 Mai} % (fold)
        \label{sub:Mardi le 9 Mai}

            % Journée du mardi le 9 mai
            \subsubsection{Changement de base} % (fold)
            \label{sec:Changement de base}
            Soit effectuons le changement de base de l'Hamiltonien de Hubbard.
            \begin{align}
                H=U\sum_{i}n_{i\uparrow}n_{i\downarrow}
                -\sum_{<i,j>,\sigma}t_{ij}\qty(
                    c^\dagger_{i,\sigma}c_{j,\sigma}
                    +
                    c^\dagger_{j,\sigma}c_{i,\sigma}
                )
            \end{align}
            Les transformées de Fourrier des opérateurs création et
            annihilation sont
            \begin{align}
                c^\dagger_{i,\bm{k}}&=\frac{1}{\sqrt{N}}\sum_{\bm{r}}
                e^{i\bm{k}\cdot\bm{r}}
                c^\dagger_{i,\bm{r}}\\
                c^\dagger_{i,\bm{r}}&=\frac{1}{\sqrt{N}}\sum_{\bm{k}}
                e^{-i\bm{k}\cdot\bm{r}}
                c^\dagger_{i,\bm{r}}
            \end{align}
            Allons-y terme à terme.

            % subsubsection Changement de base (end)
            \subsubsection{Hubbard Intro Old} % (fold)
            \label{sec:Hubbard Intro Old}
            % Problème 1:
            \begin{problem}
                Supposons que $N=2$.
            Construire une représentation matricielle
            explicite, dans l'espace de Hilbert global
            de dimension $2^2=4$, des opérateurs suivants:
            $c_1,c_2,n_1,n_2$
            \end{problem}

            % Problème 2:
            \begin{problem}
                Supposons que $N=2$ et
            considérons un état général à un électron
            $\ket{\psi}=\alpha\ket{1}+\beta\ket{2}$,
            où $\abs{\alpha}^2+\abs{\beta}^2=1$. Quelle
            est la probabilité que l'électron soit sur
            l'ion no $1$? Quelles sont les valeurs moyennes
            de $n_1$ et $n_2$?
            \end{problem}

            % Problème 3:
            \begin{problem}
                Considérer un modèle
            à $N=4$ sites et comportant $M=4$ électrons,
            avec $t=0$. Trouver les deux niveaux d'énergie
            les plus bas avec les états correspondants.
            Combien y en a-t-il pour chacun des deux
            niveaux?
            \end{problem}

            % Problème 4:
            \begin{problem}
                Démontrer la relation 22
            et ensuite que les deux équations de 20 sont
            compatibles.
            \end{problem}

            % Problème 5:
            \begin{problem}
                Démontrer que les opérateurs
                $\widetilde{c_\sigma}(\bm{k})$ et
                $\widetilde{c_\sigma}(\bm{k})$ satisfont
                aux relations d'anticommutation suivantes:
                \begin{align}
                    &\{\widetilde{c}_\sigma(\bm{k}),
                    \widetilde{c}_{\sigma'}(\bm{k'})\}=0
                    &&\{\widetilde{c}_\sigma(\bm{k}),
                    \widetilde{c}_{\sigma'}^\dagger(\bm{k'}
                    )\}=\delta_{\sigma\sigma'}
                    \delta_{\bm{k}\bm{k'}}
                \end{align}
            \end{problem}

            %Problème 6:
            \begin{problem}
                Démontrer que l'opérateur $K$ s'exprime
                comme suit en fonction des opérateurs
                $\widetilde{c_\sigma}(\bm{k})$ et
                $\widetilde{c_\sigma}^\dagger(\bm{k})$:
                \begin{align}
                    &K=-2t\sum_{\bm{k}\sigma}
                    \cos(k)n_\sigma(\bm{k})
                    &&n_\sigma(\bm{k})\equiv
                    \widetilde{c}_\sigma^\dagger(\bm{k})
                    \widetilde{c}_\sigma(\bm{k})
                \end{align}
            \end{problem}

            %Problème 7:
            \begin{problem}
                Démontrer que les opérateurs de nombre
                $n_\sigma(\bm{k})$ associés à des spins ou
                des nombres d'ondes différents commutent.
            \end{problem}

            %Problème 8:
            \begin{problem}
                Quel est l'état fondamental d'un système de
                $N$ électrons installés sur un anneau de
                $N$ sites? On est à demi-remplissage, car

            \end{problem}

            % subsubsection Hubbard Intro Old (end)


        % subsection Mardi le 9 Mai (end)

    % section Semaine 2 (end)
\end{document}
