\documentclass{subfiles}[../main.tex]

\begin{document}
    \section{Semaine 1} % (fold)
    \label{sec:Semaine 1}

        \subsection{Vendredi le 5 mai} % (fold)
        \label{sub:Vendredi le 5 mai}
            Aujourd'hui, je m'intéresse au Pfaffian
            \subsubsection{Pfaffian} % (fold)
            \label{ssub:Pfaffian}
                Le Pfaffian est non-nul pour des matrices anti-symétriques de
                dimension paires seulement. En effet
                \begin{align}
                    \det\qty(A)&=\pfaf\qty(A)^2\qquad\text{Def.}
                \end{align}
                où $A=-A^{T}$. On peut montrer que le pfaffian est nul pour une
                matrice de dimension impaire.
                \begin{align}
                    \det\qty(A)&=\det\qty(A^T)\\
                    &=\det\qty(-A)\\
                    &=(-1)^{\dim(A)}\det(A)\\
                    \Rightarrow\ \det(A)=\pfaf(A)^2&=0\quad\text{si}\ \dim(A)\ \text{impaire}
                \end{align}
                Le corollaire de cette preuve est que le déterminant d'une matrice
                anti-symétrique est toujours positif. Quelques propriétés à
                garder en tête.
                \begin{align}
                    \pfaf\qty(BAB^T)&=\det(B)\pfaf(A),\qquad\forall B\in M_{2n}\\
                    \pfaf\qty(A^{2m+1})&=(-1)^{nm}\pfaf\qty(A)^{2m+1}\\
                    \frac1{\pfaf(A)}\pdv{\pfaf(A)}{x_i}&=
                    \frac12\tr\qty(A^{-1}\pdv{A}{x_i})
                \end{align}
                Queleques autres propriétés qui peuvent être intéressante pour
                réduire la dimension du pfaffian à calculer
                \begin{align}
                    \pfaf\qty(
                    \begin{matrix}
                        A_1&0\\0&A_2
                    \end{matrix}
                    )&=\pfaf\qty(A_1)\pfaf\qty(A_2)\\
                    \pfaf\qty(
                    \begin{matrix}
                        0&M\\ M^T&0
                    \end{matrix}
                    )&=(-1)^{n(n-1)/2}\det(M)
                \end{align}
                Il existe une généralisation pour des matrices de dimension impaire,
                où on ajoute des $1$ et des $-1$ à la dernière ligne et colonne.
                Le pfaffian est donc bien défini pour cette nouvelle matrice.
                Cette définition n'est pas très utile pour le projet. Quelques
                autres propriétés et une bibliothèque peuvent être trouvés dans
                la publication de \textit{Pfapack}\cite{pfapack}.

            % subsubsection Pfaffian (end)

        % subsection Vendredi le 5 mai (end)

    % section Semaine 1 (end)
\end{document}
